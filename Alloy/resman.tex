\documentclass{article}
\usepackage{zed-csp}
\title{Illustrating modular refinement}
\author{David Faitelson}
\begin{document}
\maketitle

\section{Introduction}

This document describes a development of a simple resource manager in a
modular style. I have selected an extremely simple system so that the
modular development technique will be as clear as possible.

The development is bottom-up. We start with primitive systems of sets
and chains. Then we prove that a chain system refines a set system. Next
we use these subsystems to construct two descriptions of the resource
manager. An abstract version using sets, and a concrete version using
chains.

Because we construct both systems by using compatible operations and
because we have shown that chains refine sets we do not have to prove
that the concrete resource manager refines the abstract resource manager.

The chain subsystem is based on a similar chain described in the book
Using Z. However it is not an exact copy, most notably we do not use a
state variable to hold the last item in the chain, and the development
of the remove operation is more direct (we do not provide a pop operation
and thus do not use it when describing the remove operation).

\section{Sets}

Let $ID$ be a non empty finite set of natural numbers. We will reserve $0$ for
our own purposes. Thus we must ensure that it is available in $ID$
but we will never accept it as a valid resource identifier in our sets.

\begin{axdef}
	ID : \finset \nat
\where
	0 \in ID 
\end{axdef}

A set subsystem has a set of (non-zero) resource identifiers:

\begin{schema}{Set}
	items: \finset ID 
\where
	0 \notin items
\end{schema}

\begin{axdef}
	Set\_items : Set \fun \finset ID
\where
	\forall s : Set @ \\
\t1 		Set\_items~s = s.items 
\end{axdef}

We may initialize a set either to be empty or to contain all the
identifiers available to the system.

\begin{schema}{Set\_InitEmpty}
	Set' 
\where
	items' = \emptyset
\end{schema}

\begin{schema}{Set\_InitFull}
	Set'
\where
	items' = ID \setminus \{ 0 \}
\end{schema}

We may add or remove identifiers from the set:

\begin{schema}{Set\_Add}
	\Delta Set \\
	item?: ID
\where
	items' = items \cup \{ item? \} 
\end{schema}

\begin{schema}{Set\_Remove}
	\Delta Set \\
	item?: ID
\where
	item? \neq 0 \\
	items' = items \setminus \{ item? \} 
\end{schema}

Sometimes the set must remain the same:

\begin{schema}{Set\_Skip}
	\Xi Set
\end{schema}

\section{Chains}

A chain is a representation of sets using linked lists. We use 0 to
indicate an empty chain.

\begin{schema}{Chain}
	first : ID \\ next : ID \finj ID
\where
	first = 0 \implies next = \emptyset \\ next \rres \{ first \}
	= \emptyset \\ next \neq \emptyset \implies \\
\t1		\ran (\{ first \} \dres next \star) = \dom next \cup
\ran next \cup \{ first \} \setminus \{ 0 \} \end{schema}

\begin{schema}{Chain\_Skip}
	\Xi Chain
\end{schema}

\subsection{Initialization}

A chain may be initialized with no identifiers or with all the identifiers in the system.

\begin{schema}{Chain\_InitEmpty}
	Chain'
\where
	first' = 0 \\
	next' = \emptyset 
\end{schema}

\begin{axdef}
	Chain\_items : Chain \fun \finset ID
\where
	\forall c : Chain @ \\
\t1 		Chain\_items~c = \dom c.next \cup \ran c.next \cup \{ c.first \} \setminus \{ 0 \}
\end{axdef}

\begin{schema}{Chain\_InitFull}
	Chain'
\where
	Chain\_items~\theta Chain' = ID \setminus \{ 0 \}
\end{schema}

\subsection{Adding items}

To add an identifier to a chain we need to consider two cases. When the chain is empty we set $first$ to the requested identifier but don't change $next$. 

\begin{schema}{Chain\_AddWhenEmpty}
	\Delta Chain \\
	item? : ID
\where
	item? \neq 0 \\
	first = 0 \\
	first' = item? \\
	next' = next
\end{schema}


When the chain has one or more elements we make the new item stand at the front and link it to the previous first element. 

\begin{schema}{Chain\_AddWhenNotEmpty}
	\Delta Chain \\
	item? : ID 
\where
	item? \neq 0 \\
	first \neq 0 \\
	first' = item? \\
	next' = next \cup  \{ item? \mapsto first \}
\end{schema}

Thus adding an item to the chain is either adding it to an empty chain or to a nnon empty chain:

\begin{zed}
Chain\_Add \defs Chain\_AddWhenEmpty \lor Chain\_AddWhenNotEmpty
\end{zed}

\subsection{Removing items}

Removing an item from a chain requires considering several distinct cases. 

First, the item to be removed may be the single item in the chain. In this case the chain becomes empty. 

\begin{schema}{Chain\_Remove\_Singleton}
	\Delta Chain \\
	item? : ID 
\where
	item? \neq 0 \\
	first = item? \\
	next = \emptyset \\
	first' = 0 \\
	next' = next 
\end{schema}

Next, the item may be the first in a non empty chain.

\begin{schema}{Chain\_Remove\_First}
	\Delta Chain \\
	item? : ID 
\where
	item? \neq 0 \\
	first = item? \\
	next \neq \emptyset \\
	first' = next~first \\
	next' = \{ item? \} \ndres next
\end{schema}

Next, the item may be in the middle of a non empty chain. That is,
there will be at least one item before and one item after. In this case
we must link the item before to the item after.  

\begin{schema}{Chain\_Remove\_Middle}
	\Delta Chain \\
	item? : ID
\where
	item? \neq 0 \\
	item? \in \dom next \\
	item? \in \ran next \\

	first' = first \\
	next' = \{ item? \} \ndres next \nrres \{ item? \} \cup \{ next\inv ~ item? \mapsto next~item? \} 
\end{schema}

Finally, we may remove the last item in the chain.

\begin{schema}{Chain\_Remove\_Last}
	\Delta Chain \\
	item? : ID 
\where
	item? \neq 0 \\
	item? \notin \dom next \\
	item? \in \ran next \\

	first' = first \\
	next' = next \nrres \{ item? \}
\end{schema}

Finally, to remove an item we combine all of the possible cases:

\begin{zed}
	Chain\_Remove \defs Chain\_Remove\_Singleton \lor {} \\
\t1		Chain\_Remove\_First \lor {} \\
\t1		Chain\_Remove\_Middle \lor {} \\
\t1		Chain\_Remove\_Last 
\end{zed}

\section{Refinement}

We show that $Chain$ refines $Set$ using the following functional retrieve relation

\begin{schema}{Retrieve}
	Set \\
	Chain
\where
	Set\_items~\theta Set = Chain\_items~\theta~Chain
\end{schema}

It is clear that $Retrieve$ is a function from chains to sets because
$Chain\_items$ is a function.

\subsection{$Set\_Add$ simulates $Chain\_Add$}

To prove that $Chain\_Add$ refines $Set\_Add$ we consider two cases. First, when the chain is empty $next$ is the empty set and we argue:

\begin{argue}
	\pre A\_Add \land \Delta Retrieve \land C\_Add \\
\implies item? \neq 0 \land \{ 0 \} \setminus \{0\} = items \land \{ item? \} \setminus \{ 0\} = items' \\
\implies items = \emptyset \land items' = \{item?\} \\
\implies items' = items \cup \{ item? \} \\
= A\_Add 
\end{argue}

Next, when the chain is not empty we argue:

\begin{argue}
	\pre A\_Add \land \Delta Retrieve \land C\_Add \\
\implies item? \neq 0 \land \dom next \cup \ran next \cup \{ first \} \setminus \{ 0 \} = items \land {} \\
\t1 \dom (next \cup \{ item? \mapsto first \}) \cup {} \\
\t1 \ran (next \cup \{ item? \mapsto first \}) \cup \{ item? \} \setminus \{0\} = items' \\
\implies \dom next \cup \ran next \cup \{ first , item? \} = items' \\
\t1 \implies \dom next \cup \ran next \cup \{ first \} \setminus \{ 0 \} \cup \{ item? \} = items' \\
\implies items \cup \{item?\} = items'\\
= A\_Add
\end{argue}

\subsection{$Set\_Remove$ simulates $Chain\_Remove$}

We split the proof into four cases, following the structure of $Chain\_Remove$.  
\subsubsection{Removing the item of a one item chain}

\begin{argue}
	\pre Set\_Remove \land \Delta Retrieve \land Chain\_Remove\_Singleton \\ 
\implies item? \neq 0 \land item? = first \land \{ first \} \setminus \{0\} = items \land {} \{ 0 \} \setminus \{ 0\} = items' \\
\implies items' = items \setminus \{ item? \} \\
\implies Set\_Remove
\end{argue}

\subsubsection{Removing the first item}

\begin{argue}
	\pre Set\_Remove \land \Delta Retrieve \land Chain\_Remove\_First \\
\implies item? \neq 0 \land item? = first \land \dom next \cup \ran next \cup \{ first \} = items \land {} \\
\t1 \dom (\{item?\} \ndres next) \cup \ran (\{item?\} \ndres next) \cup \{ next ~first \} = items' \\
\implies \dom next \setminus \{ first \} \cup \ran next \setminus \{ next~first\} \cup \{ next~first\}  = items' \\
\implies \dom next \cup \ran next \setminus \{ first \} = items' & $first \notin \ran next$ \\
\implies items' = items \setminus \{ first \} \\
\implies Set\_Remove
\end{argue}

\subsubsection{Removing an item in the middle}

\begin{argue}
	\pre Set\_Remove \land \Delta Retrieve \land Chain\_Remove\_Middle \\ 
\implies item? \neq first \land item? \in \dom next \land item? \in \ran next \land {} \\
\t1  \dom next \cup \ran next \cup \{ first \} = items \land {} \\
\t1	\dom next \setminus \{ item? \} \cup \ran next \setminus \{item?\} \cup \{ first \} = items' \\
\implies items' = items \setminus \{ item? \} 
\end{argue}

The last equality holds because $first \neq item?$. 

\subsubsection{Removing the last item}

\begin{argue}
	\pre Set\_Remove \land \Delta Retrieve \land Chain\_Remove\_Last \implies Set\_Remove \\
\implies item? \in \dom next \land item? \notin \ran next \land {} \\
\t1 	\dom next \cup \ran next \cup \{ first \} = items \land {} \\
\t1 	\dom (next \nrres \{item?\}) \cup \ran (next \nrres \{item?\}) \cup \{ first \} = items' \\
\implies \dom (next \setminus \{ next\inv~item? \}) \cup {} \\
\t1 \ran (next \ setminus \{ next\inv~item? \}) \cup \{first\} = items' \\
\implies \dom next \setminus \{ next\inv item?\} \cup \ran next \setminus \{ item? \} \cup \{first\} = items' \\
\implies \dom next \cup \ran next \setminus \{item?\}\cup \{first\} = items' \\
 & $next\inv~item? \in \ran next$ \\
\implies (\dom next \cup \ran next \cup \{first\}) \setminus \{item?\}  = items' \\
& $first \neq item? \land item? \notin \dom next$ \\
\implies items' = items \setminus \{ item \}

\end{argue}

\section{An abstract resource manager}

\begin{schema}{A\_RM}
	used : Set \\
	free : Set 
\where
	Set\_items~used \cap Set\_items~free = \emptyset
\end{schema}

To use the operations we have defined for the $Set$ subsystem we define
two selection schemas.

\begin{schema}{Select\_Used}
	\Delta A\_RM \\
	\Delta Set 
\where
	used = \theta Set \\
	used' = \theta Set' \\
\end{schema}

\begin{schema}{Select\_Free}
	\Delta A\_RM \\
	\Delta Set 
\where
	free = \theta Set \\
	free' = \theta Set'
\end{schema}
	

\begin{zed}
	Used\_Init \defs \exists \Delta Set @ Select\_Used \land Set\_InitEmpty
\end{zed}

\begin{zed}
	Used\_Add \defs \exists \Delta Set @ Select\_Used \land Set\_Add
\end{zed}

\begin{zed}
	Used\_Remove \defs \exists \Delta Set @ Select\_Used \land Set\_Remove
\end{zed}

\begin{zed}
	Used\_Skip \defs \exists \Delta Set @ Select\_Used \land Set\_Skip
\end{zed}

\begin{zed}
	Free\_Init \defs \exists \Delta Set @ Select\_Used \land Set\_InitFull
\end{zed}

\begin{zed}
	Free\_Add \defs \exists \Delta Set @ Select\_Free \land Set\_Add
\end{zed}

\begin{zed}
	Free\_Remove \defs \exists \Delta Set @ Select\_Free \land Set\_Remove
\end{zed}

It is easy to see that because $Used\_$ and $Free\_$ operate on separate state variables, they are compatible.

\subsection{Initialization}

\begin{schema}{A\_Init}
	\Delta A\_RM 
\where
	Used\_Init \\
	Free\_Init
\end{schema}

\subsection{Operations}

\begin{schema}{A\_Acquire}
	\Delta A\_RM \\
	item? : ID 
\where
	Used\_Add \\
	Free\_Remove
\end{schema}

\begin{schema}{A\_Release}
	\Delta A\_RM \\
	item? : ID 
\where
	Used\_Remove \\
	Free\_Add
\end{schema}

\begin{schema}{A\_Dispose}
	\Delta A\_RM \\
	item? : ID
\where
	Used\_Skip \\
	Free\_Remove
\end{schema}

\begin{schema}{A\_Purchase}
	\Delta A\_RM \\
	item? : ID
\where
	Used\_Skip \\
	Free\_Add
\end{schema}

\section{A concrete resource manager}

\begin{schema}{C\_RM}
	used : Chain \\
	free : Chain 
\where
	Chain\_items~used \cap Chain\_items~free = \emptyset
\end{schema}

To use the operations we have defined for the $Chain$ subsystem we define
two selection schemas.

\begin{schema}{C\_Select\_Used}
	\Delta C\_RM \\
	\Delta Chain 
\where
	used = \theta Chain \\
	used' = \theta Chain' \\
\end{schema}

\begin{schema}{C\_Select\_Free}
	\Delta C\_RM \\
	\Delta Chain 
\where
	free = \theta Chain \\
	free' = \theta Chain'
\end{schema}
	
\begin{zed}
	C\_Used\_Init \defs \exists \Delta Chain @ C\_Select\_Used \land Chain\_InitEmpty
\end{zed}

\begin{zed}
	C\_Used\_Add \defs \exists \Delta Chain @ C\_Select\_Used \land Chain\_Add
\end{zed}

\begin{zed}
	C\_Used\_Remove \defs \exists \Delta Chain @ C\_Select\_Used \land Chain\_Remove
\end{zed}

\begin{zed}
	C\_Used\_Skip \defs \exists \Delta Chain @ C\_Select\_Used \land Chain\_Skip
\end{zed}

\begin{zed}
	C\_Free\_Init \defs \exists \Delta Chain @ C\_Select\_Used \land Chain\_InitFull
\end{zed}

\begin{zed}
	C\_Free\_Add \defs \exists \Delta Chain @ C\_Select\_Free \land Chain\_Add
\end{zed}

\begin{zed}
	C\_Free\_Remove \defs \exists \Delta Chain @ C\_Select\_Free \land Chain\_Remove
\end{zed}

\subsection{Initialization}

\begin{schema}{C\_Init}
	\Delta C\_RM 
\where
	C\_Used\_Init \\
	C\_Free\_Init
\end{schema}

\subsection{Operations}

\begin{schema}{C\_Acquire}
	\Delta C\_RM \\
	item? : ID 
\where
	C\_Used\_Add \\
	C\_Free\_Remove
\end{schema}

\begin{schema}{C\_Release}
	\Delta C\_RM \\
	item? : ID 
\where
	C\_Used\_Remove \\
	C\_Free\_Add
\end{schema}

\begin{schema}{C\_Dispose}
	\Delta C\_RM \\
	item? : ID
\where
	C\_Used\_Skip \\
	C\_Free\_Remove
\end{schema}

\begin{schema}{C\_Purchase}
	\Delta C\_RM \\
	item? : ID
\where
	C\_Used\_Skip \\
	C\_Free\_Add
\end{schema}

\section{Conclusion}

The concrete version of the resource manager has the exact same
structure as the abstract version, indeed with a simple tool it can
be automatically generated from the abstract description. The concrete
version is a refinement of the abstract version because the kernel
operations of both versions are compatible and because the kernel
operations of the concrete version simulate their corresponding abstract
versions. The proof effort is concentrated in showing that $Chain$
refines $Set$. Once we have this building block in place we can use it
to construct the system with a minimal proof effort, as proving
compatibility reduces to observing that the operations act on separate
state variables.

\end{document}
